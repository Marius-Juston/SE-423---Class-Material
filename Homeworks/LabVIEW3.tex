\documentclass[11pt]{article}
% Required packages:
% \textbf{
% This Is An Optional Assignment If You Would Like To Learn More About Graphs And Plotting In LABVIEW.
% }

\def\AssignmentMode{LabVIEW}     % HW | Lab | LabVIEW
\def\HWNum{3}
\def\AssignmentText{This Is An Optional Assignment If You Would Like To Learn More About Graphs And Plotting In LABVIEW.}

% Course metadata
% ============================
% Required packages
% ============================

% \usepackage[letterpaper, total={6in, 9.5in}]{geometry}
\usepackage[
  letterpaper,
  left=0.5in,
  right=0.5in,
  top=0.5in,
  bottom=1.0in
]{geometry}


\usepackage[most]{tcolorbox}
\usepackage{booktabs}
\usepackage{fancyhdr}
\usepackage{graphicx}
\usepackage[
  colorlinks=true,
  linkcolor=blue,
  urlcolor=blue,
  citecolor=blue
]{hyperref}
\usepackage{lastpage}
\usepackage{listings}
\usepackage{minted}
\usepackage{setspace}
\usepackage{subcaption}
\usepackage{tabularx}
\usepackage{tikz}
\usepackage{xcolor}
\usepackage{xparse}
\usepackage[siunitx, american, RPvoltages]{circuitikz}
% Required packages:
\usepackage[table]{xcolor}
\usepackage{array}
\usepackage{array,colortbl,xcolor}
\usepackage{pdflscape}
\usepackage{pgfplots}
\usepackage{xurl}
\usetikzlibrary{arrows.meta, angles, quotes, shapes.symbols, arrows.meta}


\usepackage{pgfcalendar}

%%%%%%%%%%%%%%%%%%%% MODIFY DATES HERE %%%%%%%%%%%%%%%%

\def\Year{2026}
\def\StartDate{\Year-01-19}

\def\HWCheckOffTime{5PM}
\def\HWSubmitTime{9AM}
\def\LabVIEWSubmitTime{5PM}

%%%%%%%%%%%%%%%%%%%%%%%%%%%%%%%%%%%%%%%%%%%%%%%%%%%%%%%

\newcount\julianday
\newcount\myweekday % Define a counter for the weekday index

\newcommand{\GetDate}[2]{%
    \pgfcalendardatetojulian{\StartDate}{\julianday}%
    \advance\julianday by \numexpr #1*7 + #2\relax%
    % 1. Convert JDN to a date (sets \myyear, \mymonth, \myday)
    \pgfcalendarjuliantodate{\julianday}{\myyear}{\mymonth}{\myday}%
    % 2. Convert JDN to a weekday index 0..6 and store in \myweekday
    \pgfcalendarjuliantoweekday{\julianday}{\myweekday}%
    % 3. Print the name using the index, then the rest
    \pgfcalendarweekdayname{\myweekday}, \pgfcalendarmonthname{\mymonth} \myday%
}

\def\Final{\GetDate{16}{3}}
\usepackage{tikz}
\usepackage{xcolor}

\definecolor{iconGold}{RGB}{249,219,132}
\definecolor{iconGreen}{RGB}{71,160,93}

% Derived Border Colors (Darkened for contrast)
\definecolor{borderGold}{RGB}{180,150,80}
\definecolor{borderGreen}{RGB}{50,110,65}

\tikzset{
    ide_icon/.style={
        baseline=-0.3ex,
        line join=round,
        line cap=round,
        % Set a default border width that scales 
        line width=0.4pt 
    }
}

\newcommand{\pauseBtn}[1][0.4]{%
\begin{tikzpicture}[ide_icon, scale=#1]
    \filldraw[fill=iconGold, draw=borderGold] (0,0) rectangle (0.25, 0.8);
    \filldraw[fill=iconGold, draw=borderGold] (0.4,0) rectangle (0.65, 0.8);
\end{tikzpicture}}

\newcommand{\playBtn}[1][0.4]{%
\begin{tikzpicture}[ide_icon, scale=#1]
    \filldraw[fill=iconGold, draw=borderGold] (0,0) rectangle (0.15, 0.8);
    \filldraw[fill=iconGreen, draw=borderGreen] (0.25,0) -- (0.25,0.8) -- (0.75,0.4) -- cycle;
\end{tikzpicture}}

\newcommand{\debugBtn}[1][0.55]{%
\begin{tikzpicture}[ide_icon, scale=#1, rotate=45]
    % Legs: use border color for the stroke
    \draw[borderGreen, line width=0.6pt] (-0.3, 0.5) -- (0.3, 0.1);
    \draw[borderGreen, line width=0.6pt] (-0.3, 0.3) -- (0.3, 0.3);
    \draw[borderGreen, line width=0.6pt] (-0.3, 0.1) -- (0.3, 0.5);

    % Antenna
    \draw[borderGreen, line width=0.6pt] (0, 0.6) -- (-0.15, 0.75);
    \draw[borderGreen, line width=0.6pt] (0, 0.6) -- (0.15, 0.75);
    % Body
    \filldraw[fill=iconGreen, draw=borderGreen] (0,0.3) ellipse (0.18 and 0.25);
    % Head
    \filldraw[fill=iconGreen, draw=borderGreen] (0,0.6) circle (0.08);

\end{tikzpicture}}

\newcommand{\restartBtn}[1][0.8]{%
\begin{tikzpicture}[ide_icon, scale=#1]

    % Shaft with border
    \draw[borderGold, line width=1.8pt]
        (0,0.2) -- (0,0) -- (0.4,0) -- (0.4,0.2);

    \draw[iconGold, line width=1.0pt]
        (0,0.2) -- (0,0) -- (0.4,0) -- (0.4,0.2);

    % Arrowhead with proper outline
    \filldraw[
        fill=iconGold,
        draw=borderGold,
        line join=round
    ]
        (-0.1,0.2) -- (0.1,0.2) -- (0,0.33) -- cycle;

    % Green triangle cap
    \filldraw[fill=iconGreen, draw=borderGreen]
        (0.15,0.1) -- (0.15,0.5) -- (0.4,0.3) -- cycle;

\end{tikzpicture}%
}

% ============================
% Assignment mode resolution
% ============================

\def\CourseCode{SE 423}
\def\CourseName{Mechatronics}

\def\AssignmentType{}
\def\AssignmentFooter{}

\makeatletter
\@namedef{Assignment@HW@type}{Homework Assignment}
\@namedef{Assignment@HW@footer}{HW}

\@namedef{Assignment@Lab@type}{Laboratory Assignment}
\@namedef{Assignment@Lab@footer}{Lab}

\@namedef{Assignment@LabVIEW@type}{LabVIEW Assignment}
\@namedef{Assignment@LabVIEW@footer}{LabVIEW}

\@ifundefined{Assignment@\AssignmentMode @type}{
  \PackageError{AssignmentMode}{Invalid AssignmentMode}{%
    Valid options: HW, Lab, LabVIEW}
}{
  \edef\AssignmentType{\@nameuse{Assignment@\AssignmentMode @type}}
  \edef\AssignmentFooter{\@nameuse{Assignment@\AssignmentMode @footer}}
}
\makeatother

% ============================
% Header / footer
% ============================

\pagestyle{fancy}
\fancyhf{}
\fancyfoot[L]{// \CourseCode, \CourseName}
\fancyfoot[R]{// \AssignmentFooter~\HWNum, Page \thepage\ of \pageref{LastPage}}
\renewcommand{\headrulewidth}{0pt}

% ============================
% Title box
% ============================

\newcommand{\MakeAssignmentTitle}{
\begin{tcolorbox}[
    colback=gray!30!white,
    width=\textwidth,
    boxrule=0.8pt,
    arc=2mm,
    left=6mm,
    right=6mm,
    top=4mm,
    bottom=4mm,
    center
]
\centering
\setstretch{1.2}

\textbf{\Large \CourseCode\ \CourseName \\ \AssignmentType\ \#\HWNum}

\vspace{2mm}

\textbf{
\AssignmentText
}
\end{tcolorbox}
\vspace{0.5cm}
}

\setmintedinline{breaklines, breakafter=_}


\setminted{
  breaklines,
  breakanywhere=false,
  breakafter=\space,
  frame=lines,
  fontsize=\footnotesize,
}


\lstset{
    basicstyle=\ttfamily,   % Typewriter font for code
    breaklines=true,        % Allow breaking long lines
    columns=flexible        % Makes spacing better for inline
}

\newcounter{exercise}

\NewDocumentCommand{\Ex}{o}{
  \stepcounter{exercise}
  \section*{Exercise \theexercise
    \IfValueT{#1}{: #1} 
  }
  \addcontentsline{toc}{section}{Exercise \theexercise
    \IfValueT{#1}{: #1}}
}


\newcolumntype{C}[1]{>{\centering\arraybackslash}p{#1}}

\pgfplotsset{compat=1.18} 

\begin{document}

\MakeAssignmentTitle

\Ex

Demonstrate your programs working and be prepared to show me how you built parts of your LabVIEW program. Grading for this assignment is simply full credit if you did the assignment and no credit if you did not complete the assignment. Make sure to ask your instructors questions if you get stuck.    

I would like you to go through the “Basics of Control Design and Simulation” tutorial. It can be found at \url{http://coecsl.ece.illinois.edu/se423/NI-Tutorial-BasicsofCDS.pdf}. When you are done with this tutorial you will have saved three files:  “\lstinline|My Controls Example VI.vi|”, “\lstinline|My Controls Example With Math Script VI.vi|” and “\lstinline|My Control and Simulation Example VI.vi|”  \textbf{Demo all three of these VIs working for the first part of your check off. }

The last VI you built “\lstinline|My Control and Simulation Example VI.vi|” uses the Control and Simulation Loop. You should be familiar with MathWorks Simulink from previous classes like SE320 and ME360. You can think of the blocks you place inside the Control and Simulation Loop in the same way as blocks in a Simulink simulation.  

For this next exercise, I would like you to reproduce the VI shown in the pictures below. For this exercise, start by creating a LabVIEW Project. Then, inside this project, add a single VI. The reason I want you to use a project is that I would like you to try out a “shared variable” created in a separate \verb|.lvlib| file that will also become part of your project. I will give some pointers and instructions below, but they will not necessarily be in the exact order you need to perform the steps or complete the steps. In other words, you may need to do a small amount of your own hunting to find the necessary blocks and other parts of the program, and use LabVIEW help to figure out how to use/wire certain blocks. Many of the blocks I used below come from the “Control Design \& Simulation $\rightarrow$ Simulation” section of LabVIEW. 
%
\begin{figure}[H]
    \centering
    \includegraphics[width=1\linewidth]{Homeworks/Figures/LabVIEW3/Picture1.png}
\end{figure}
%
\begin{figure}[H]
    \centering
    \includegraphics[width=1\linewidth]{Homeworks/Figures/LabVIEW3/Picture2.png}
\end{figure}
%
Pointers/Instructions
%
\begin{enumerate}
    \item Why am I using two Control and Simulation Loops? I am splitting the simulation into two parts. One to run the simulation and the second to plot the data. Mainly to show you how to use global “Shared Variables”. The simulation’s “y” output is stored in a “Shared Variable” with Real-Time FIFO enabled. In the block diagram view below, it is the orange “SaveData” block. We will set up one simulation loop to run on one processor core and the other to run on another. 
    \item After you add the two simulation loops, double click on their parameters tab in their top left corner, which will bring up the Configure Simulation Parameters window. Set up as in the pictures below. For the second Simulation Loop, set the same, except for the Processor Assignment: set it to “1”. 
\begin{figure}[H]
\centering
\begin{subfigure}[b]{0.485\linewidth}
\centering
    \includegraphics[width=1\linewidth]{Homeworks/Figures/LabVIEW3/Picture3.png}
% \caption{First image}
\end{subfigure}
\begin{subfigure}[b]{0.485\linewidth}
\centering
    \includegraphics[width=1\linewidth]{Homeworks/Figures/LabVIEW3/Picture4.png}
% \caption{First image}
\end{subfigure}
\end{figure}
\item In your project window, right-click on the “My Computer” under your project name and select New$\rightarrow$Variable. Give the variable a name. Make the variable type “Single Process” and the data type “Double”. Click on the RT FIFO item and check the item “Enable RT FIFO”. The shared variable is stored in a library \lstinline|.lvlib| file. Save this library file with a name of your choosing.
\item To add the shared variables to your application, drag the shared variable from the project window into your block diagram window. Then, in your Block Diagram, right-click on the shared variable and select the appropriate “Access Mode.”  
\item Block Locations:
\begin{enumerate}
    \item Use the Flat Sequence to order when items of your program run. Programming $\rightarrow$ Structures.
    \item Transfer Function:  Control Design \& Simulation$\rightarrow$ Simulation $\rightarrow$ Continuous. The transfer function inside the block can be whatever you choose
    \item Signal Generator:   Control \& Simulation $\rightarrow$ Simulation $\rightarrow$ Signal Generation
Notice that I set up the Signal Generator to have a terminal input. I want the Amplitude of the Signal Generator to change randomly in each run. If you double-click on the signal generator and select the amplitude parameter, you can change its parameter source to “terminal.”  
\item Collector:  Control \& Simulation $\rightarrow$ Simulation $\rightarrow$ Utilities. The Collector is optimized for the simulation loop and collects data for later plotting, even offline. To create the Collector’s output variable that I called “signal history”, hover your mouse just to the right of the Collector and right-click. Select Create$\rightarrow$Indicator.   
\item The Waveform Chart and its accompanying block to its left is called a “SimTime Waveform”. Find it under Control \& Simulation$\rightarrow$Simulation$\rightarrow$Graph Utilities.
\item Select and Greater:  Programming $\rightarrow$ Comparison. I am showing you the Select block so you can see how to choose between two signals and route only one to a desired block. 
\item Dice (Random Number):  Programming $\rightarrow$ Numeric.
\item Local Variable:  Programming $\rightarrow$ Structures.
\item To add the XY Graph to your program, you must switch to the Front Panel window. From there, right-click and find the XY Graph in Modern $\rightarrow$ Graph.
\item Property Nodes \lstinline|XScale.Minimum|, \lstinline|XScale.Maximum|, \lstinline|YScale.Minimum|, \lstinline|YScale.Maximum|. Property Nodes allow you to change properties of a front panel object (in this case, a graph) during your program run. Here I am just setting the range of the X and Y scales. You will see there are a huge number of properties you can change on the fly if you wish. Since there are so many, it can be hard to find the correct property to change the item you are interested in. Using Help and trial and error is the way to figure out which property to change, finally. Create these property nodes by right-clicking on the Waveform Chart block and selecting Create $\rightarrow$ Property Node $\rightarrow$ XScale $\rightarrow$ Range $\rightarrow$ Minimum. Do the same for X Max, Y Min, and Y Max. If you right-click on the Property Node, the top option allows you to change the property to a “Write”. (Or “Read” if you need to switch back.) 
\end{enumerate}
\item In the Front Panel window, right-click on both of the graph objectives and turn on their different “Visible Items”. The “Graph Palette” allows you to zoom in and out of the graph’s data. The “Cursors Legend” is only available in the XY Graph, and it creates a cross-hair cursor. After you run your program, play around with both the zooming capability and the cursors.  
\end{enumerate}
%
To save your collected data to a file to load into Excel or MATLAB, right-click the plot and select Export. This allows you to export directly to an Excel file, or you can copy the data to the clipboard, then open your favorite editor (I like Notepad++) and paste all the data into that file. Play around with this export feature and make sure you can pull data into Excel or your favorite editor.

\end{document}