\documentclass[11pt]{article}

\usepackage{hyperref}
\usepackage{xcolor}

\usepackage{pgfcalendar}

%%%%%%%%%%%%%%%%%%%% MODIFY DATES HERE %%%%%%%%%%%%%%%%

\def\Year{2026}
\def\StartDate{\Year-01-19}

\def\HWCheckOffTime{5PM}
\def\HWSubmitTime{9AM}
\def\LabVIEWSubmitTime{5PM}

%%%%%%%%%%%%%%%%%%%%%%%%%%%%%%%%%%%%%%%%%%%%%%%%%%%%%%%

\newcount\julianday
\newcount\myweekday % Define a counter for the weekday index

\newcommand{\GetDate}[2]{%
    \pgfcalendardatetojulian{\StartDate}{\julianday}%
    \advance\julianday by \numexpr #1*7 + #2\relax%
    % 1. Convert JDN to a date (sets \myyear, \mymonth, \myday)
    \pgfcalendarjuliantodate{\julianday}{\myyear}{\mymonth}{\myday}%
    % 2. Convert JDN to a weekday index 0..6 and store in \myweekday
    \pgfcalendarjuliantoweekday{\julianday}{\myweekday}%
    % 3. Print the name using the index, then the rest
    \pgfcalendarweekdayname{\myweekday}, \pgfcalendarmonthname{\mymonth} \myday%
}

\def\Final{\GetDate{16}{3}}
%
\begin{document}
%
\begin{center}
    {\Large \textbf{SE 423 -- INTRODUCTION TO MECHATRONICS}}\\
    \vspace{0.2cm}
    Lecture 2
\end{center}
%
\vspace{0.5cm}

\section*{Introduction}

Any questions before we get started today?

Enjoyed the first lab?

There will be office hours with Dan today if you need help with Code Composer, the red board, or anything else!

Has everyone finished the soldering for this lab? If not, that is okay, and you will have time to finish it up during the next lab. I recommend you finish it by the next lab if you have the time, so that you can get started on the next lab immediately.

Have you had time to install Code Composer or get it working on your machine? It is not necessary, but it will help you. Especially towards the end of the semester, and help you out with the HWs, you can work on the HWs at the labs, or you can take that green and red board and work on it at home if you have it installed on your laptop.

However, there is a little inertia getting it all figured out and all that. Find Dan or me to get this setup and help you understand the different stuff that might be confusing you.

Let us talk about the due dates and make sure everyone understands them.

With labs, we will talk about that in a bit.

However, you need Lab 1 before Lab 2 starts, Lab 2 before Lab 3, and so on. Each lab will build on the others. Some labs are multi-week, and some are just one week. So, Lab 1, which is really set to start this week, is a one-week lab. Lab 1 is pretty straightforward. Lab 2 is pretty straightforward as well. 

We will explain more about what you need to turn in as well. Labs are done as a \textbf{group}. HWs are done \textbf{individually}. We will make this more straightforward in the lab this week. 

(OPEN LABVIEW ASSIGNMENT)

Also, we have the LabVIEW assignments; you are watching online videos and teaching yourself LabVIEW. The first exercises are straightforward; you can start them now. We will not be lecturing on LabVIEW. You can, of course, find us if you need help. The LabVIEW assignment is due \textcolor{red}{Thursday at 5 PM}.
The grading for LabVIEW is very straightforward. You show the exercise working; we are not going to check every detail. You will show us the assignment, and we will check it off as done for full credit!

The LabVIEW assignments are 5\% of your grade. The first one is due \textcolor{red}{February 5th at 5 PM}. We prefer you be done with LabVIEW before your lab. On Thursdays from 2 PM to 5 PM, that is the time when the Thursday can check off their LabVIEW. If you are on the Wednesday afternoon labs, they hope you get the LabVIEW check-off then or before. The idea is to get it done in your lab section that week. It is fine for the check-off to happen outside that time, but it is highly preferred for it not to be the case.

After we finish talking on Thursday, at the beginning of the lab, there is usually a precursor talk that outlines what will happen in that lab. Then we will ask, ``If anyone has LabVIEW to check off, let us do it right now". Or Thursday after the lecture.

The best thing is to get it started right now and check it off this week! It is pretty

You can get LabVIEW on your own machine through Citrix or through the UIUC WebStore. You will need to install the ``Simulation and Control modules" or something like that.

Reminder: if you are installing locally, it can take up to 4 hours.

(OPEN HW \#1)

That same week, we also have HW \# 1 due for some coding.

We will be coding the robot in the lab this week. How does the number work?

For questions 5, 7, 8, 9, and 10, each assignment needs to be checked off by me, the TA, or Dan. That means you need to come in and show us the code working. We want to see that your code works on your red/green board as you solder it up. It is not enough to take the demo board, plug it into your red board, and make sure it is working. We want to make sure YOUR board is working properly, soldered correctly, and all that. You need to do it on your own board.

It gets hectic during those times, but we will make sure that we are in the lab to get check-offs.

When you come into class on Wednesday morning, that is the first HW due date, where you submit HW \#1. There will be questions you need to answer; you do not need to print your C code. You will be submitting your C code to a Box folder for everybody that will be created this week. We give it some time for people dropping out of the class. We will also give you access to these recorded lectures. 
In the box folder, there will be one for the lectures, one for the lab stuff, and one for both.

Is that clear? It will be clearer when you start working on the assignment. 

We will be busy with due dates next! 

You guys should be getting 24-hour access. Next time you go to the office hours or the lab, scan your I-Card to make sure it works!

The first thing we will do during the lectures is ensure the due dates are precise! So be sure to ask around.

CodeComposer cannot do HW \#1, so I recommend using CodeBlock, VSCode, CLion, or whatever to complete the HW. The code will be simple, so you do not need anything too heavy.

(GO THROUGH THE SYLLABUS)

The office hours may be adjusted based on the TAs' schedules and other factors. So that is still subject to change.

Textbooks, there are none.

There are no plans for quizzes, but that can change depending on class attendance in the lecture. To help motivate people to come in, 9 AM classes are not everyone's favorite time, myself included; however, I do not want to lecture to an empty class, so that incentives might be necessary.

If there are, it would be quick 5-minute quizzes to make sure that you were listening to what was being said. Though this will be open quizzes, so that you can talk to the people around you, and please do!

The amount of work you put into this class will be graded, especially when you are doing the final project, since there are individual reflections at the end of the final project. Moreover, the instructors are in the lab most of the time, so we can quickly see who is doing the work and who is not.

So be the person who puts a lot of work, and your grade will improve for that!

Semester project! That is the most important part of this class, where you put it all together, all the stuff you have been doing all semester, going into a single program to make your robot car do the autonomous job it is supposed to do.

What precisely that autonomous job will be is still being finalized, but it will be close to last semester's.

That will be discussed later.

\href{https://coecsl.ece.illinois.edu/se423/}{(GO TO LAB WEBSITE)}

If you visit the lab website, you can see what was done in previous semesters.

You will make a website like theirs by the end of the semester. There will be a video of your robot working, collecting golf balls, and all that great stuff.

The robot is collecting the colored gold balls, while avoiding the squares and dropping them off in the corners.

(BACK TO SYLLABUS)

For that final project, we want to build your skills throughout the semester, so you become the person who writes the code, coming up with new mechanical grippers/gatherers for golf balls and that kind of stuff.

The people who are excited about the final project are the people who put in the most effort for the source code, since most of the mechanics are given to you.

The big one, DO NOT BOOK YOUR FLIGHT BEFORE \textcolor{red}{\Final} from 11:00 AM to 2:00 PM, that is the last Friday of finals week. You may have finals during that time; that is fine, and you should be able to come later if you can.

Demo day is all smiles and clapping as everyone sees how far your robot gets! It is a frustrating day if it does not work. We are really looking at how much effort you are putting in, even more than seeing it work. However, we would like to see it working. Our job is to make sure that it is working by the end of the semester.

If you have a final from 1:30-4:30 on \Final, you can leave the demo day early. We are right in the middle of the timelines. You MUST be at this final; this is your final. 

If you say ``I am just going to let my group members demo it", you will lose points. Some points are just for attending the final. So be sure to put it on your calendar!!


The labs are worth 30\%. Most of the points will come from completing the lab in the lab session. There will be some extra time for answering questions and commenting on the code, but most of the points will come from completing the lab.

We want you to be doing the labs with your partner. It makes it much more complicated if you split it half-half. You cannot check off your lab without your partner! If you have an interview or something like that, you have to schedule a time with your partner to figure out how to complete the lab.

For the homework, there are 25\% and 5\% for LabVIEW assignments, for a total of 30\%.

The semester project is very important for this class; that is where you put all of this together and get a good understanding of what you have learned this semester.

Do not cheat! Check out the policy for that!

For the topics, you are welcome to review them. The due dates are available there. With Canvas, you should be able to import the assignment calendar directly into your calendar (as long as all the assignments are appropriately set up).

( SHOW WEBSITE )

Let us show the website: there are pages for all the homework assignments, the lecture notes, and the lab handouts.

You will find all the HW assignments on that page; they are already posted, so you can get started if you want.

For Lab \# 2, it does ask you to build a little program in LabVIEW already, which is next week, so do not wait until Thursday to get started with LabVIEW, otherwise you will be very lost in the lab. So, get started on LabVIEW as soon as possible.

For GitHub, we went around last week and set up and checked off; we would consider that repository the HW repository. What we are going to do with that repository is have you clone it and use it for Labs. We will have it in the root folder on the C drive, with both your Net ID. 

The idea is that one of you will create the repository on GitHub and grant your group mate full privileges. Make sure that you actually do that step. 

(OPEN CODECOMPOSER)

Let us look at CodeComposer. When we start working with the CodeComposer, you will create multiple projects. CodeComposer does not really like network drives; we recommend you keep the software on the C drive. 

When you generate your repository, you will have a workspace folder where you will put all your work for the whole semester. The other folders are just libraries, documents, and other items you should not need to modify. If we make changes to these folders, it will be very easy to merge those changes.

You will be saving everything in the workspace.

CodeComposer is based on Eclipse, an open-source platform. Texas Instruments has decided to use it and give it its own flavor. CodeComposer is primarily used by a single user. You need to remember that when you open CodeComposer during lab, the last person who used the software will have their default folder showing in the main scene. So if you go fast in the lab, it is very easy for you go into someone else's repository and workspace.

Whenever you create new starter code or files, you will be asked to add your initials to your files to help mitigate that kind of error.

Make sure you do NOT click ``Use this as the default and do not ask again."

Make sure you launch the ``workspace" folder, and we will also go through it in the lab. 

So now we have a space. Now we want to create a new project.
We want to do Project $\rightarrow$ Import CCS Project  $\rightarrow$ Browse. 

We do this because we want to use the starting project code that we are giving you.

Go up one folder. TI provides starting code for these chips in ``C200Ware"; it puts its examples deep inside the folder.
C200Ware $\rightarrow$ device\_support $\rightarrow$ f2837xd (the chip that is on your board ) $\rightarrow$ examples $\rightarrow$ cpu1

Now we are looking for ``HWstart" or ``Labstarter"

And then we pull it in. If there are updates available, we recommend NOT updating as this causes problems when pushing to the chips.

Rename this project for the assignment ``HW1\_mj" (HW number and then your initials), the only file that you will really only change is just ``HWstart\_main.c". Now it is ready to be built! To build the project, you press the hammer. To push the code to the red board, push the bug (it will also build the code).

You will do the same thing in the lab and in homework.

Make sure that you only select CPU1! Do not have both CPUs selected.

The little blue arrow shows where the program is at right now.

\end{document}